\chapter{Conclusion}

The full Platypus tournament was originally speculated to be infeasible to
run, but I have completed in 9 days by detecting equivalent machines and
then using a fast interpreter for the remaining matches.

Simple algorithms suffice to detect many equivalent machines: dead
code elimination determines that three fifths of all possible Platypus
machines are redundant, and additionally renaming the Emu and
Wombat states of a machine makes four fifths of all Platpypus machines
redundant. A more complex algorithm like an adaptation of DFA
minimisation has diminishing returns, but stills help to reduce the
number of remaining matches (which is quadratic to the number of
remaining machines).

While equivalence detection is much faster than running all of the
remaining matches, faster equivalence detection allows for faster
experimentation with detection algorithms, and speeds up finding
counter-examples which were invaluable while testing new algorithms.
The equivalence detection can be parallelised across multiple threads,
and DFA minimisation for small machines like Platypus machines
can further be sped up using a data-parallel algorithm.

The Platypus tournament can be parallelised over multiple threads of a
modern CPU, multiple lanes of single instruction-multiple data extensions,
and the many lanes of GPUs, and over multiple computers. But care must
be taken to avoid communicating all the results, as the atomic operations
required to record a result of a match can take more time than running
the match on large CPUs and GPUs; a distributed Platypus tournament
similarly must avoid being bottlenecked by network latency when recording
the results of many matches.

\section{Future directions}

The results suggest that we could reduce the time taken to run full
tournaments of similar games by considering equivalences at a higher
level of abstraction. For example, some machines will perform identically
in a tournament while not behaving identically in each individual match,
as I found for flipping directions in the Platypus game. More redundancy
can be found by considering the possible behaviours of individual players in a
multi-player game, as I found for the first player in the Platypus game.

Further analysis of the results would be interesting, as I have only presented
patterns in the top 10 machines of the full tournament and in the top 5
machines for either player. I did not cover any trends or distributions within
across all the machines; one could find the performance of machines with
particular properties, such as if the Platypus state is reachable and thus the
machine may halt the game.

The results would also be useful to measure the accuracy of running
competitions between machines with different structures. Such competitions
involve fewer matches but might be less accurate, trading time for precision.
The rankings from the full tournament could be compared to the rankings
from some other competition to determine the accuracy of the competition.

The interpreters and efficient representation require some tricky coding to
implement, which at best requires more effort from researchers to get high
performance, and at worst deters researchers from investigating similar
games. Such deterrance could be reduced by tools to generate efficient
representations and interpreters from less tricky specifications. Tools
already exist for compiling specifications of custom representations of
algebraic data types \cite{bit-stealing}, for converting interpreters
into compilers \cite{meta-compilation}, and for automatically porting
code written for CPUs to run on GPUs \cite{gpu-first}, which may all
be relevant.