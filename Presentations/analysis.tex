\documentclass[a4paper]{article}

% fonts
\usepackage{microtype}
\usepackage{ClearSans}
\usepackage{charter}
\usepackage[tt]{algolrevived}

% layout
\usepackage[table]{xcolor}
\usepackage[margin=1in]{geometry}
\usepackage[hidelinks]{hyperref}
\usepackage{graphicx}
\usepackage{pdfpages}

% titles
\usepackage{titlesec}
\usepackage{titling}
\titleformat*{\section}{\Large\raggedright\bfseries\sffamily}
\titleformat*{\subsection}{\Large\raggedright\sffamily}
\titleformat*{\subsubsection}{\large\raggedright\sffamily}
\setcounter{secnumdepth}{0}

\begin{document}

\title{The \emph{Platypus Games Without Frontiers} slides and analysis}
\author{Hayley Patton}
\maketitle

\tableofcontents

\newpage

\section{Progress presentation 1}

\includepdf[nup=1x2, pages=-]{progress-1.pdf}

\section{Progress presentation 2}

\includepdf[nup=1x2, pages=-]{progress-2.pdf}

\section{Final presentation}

\includepdf[nup=1x2, pages=-]{final.pdf}

\section{Analysis of learning outcomes}

\subsection{Creative thinking}

The project was based on the theory by James Harland that it would
be infeasible to run the Platypus game, whereas I had speculated that
it would be possible, but had no evidence towards it being possible. It
was only until the end of the project, when using cloud computing,
that a schedule for successfully completing the project could be built
around the projected time needed to run the tournament.

The SIMD-based interpreter required much effort to implement, and I
did not have any evidence that it would be faster than the original
na\"ive interpreter.

\subsection{Responsibility}

Due to the limited audience of the project, the main ethical issue I
identified was the energy usage of running the Platypus tournament,
which could be large for the narrow utility derived from the results of the
tournament. I found that performance optimisations would consistently
reduce the projected energy usage with my desktop; I cannot measure the
energy used in cloud computing, but AWS offers newer GPUs which I assume
have better energy efficiency than my older desktop GPU.

\subsection{Critical thinking}

I had initially assumed that having lanes be idle in the SIMD-based
interpreter would be detrimental to performance, would the interpreter
wait for matches in all lanes to finish. Later benchmarks indicated that
refilling any lane when it finishes was 15\% to 32\% faster which is
still a speedup, but smaller than what I assumed.

My results for energy efficiency of the na\"ive (scalar) and SIMD interpreters
conflicted with published results on energy efficiency of SIMD code, but the
published results were for a different processor micro-architecture and with
different (but more precise) power measurement instruments.

Analysis of the results of the full Platypus tournament indicated that future
users of the methodology (of equivalence detection and brute forcing on cloud
GPUs) could further benefit from using an appropriate definition of
equivalence. I didn't any obvious equivalences between the top 10 machines,
but I did miss various phenomena which cause machines to perform
identically in the tournament despite behaving differently.

\subsection{Problem solving}

Multiple equivalence detection algorithms and interpreters were created
during the project, but some algorithms were strictly better and so there
were hardly any tradeoffs between the alternatives. The faster interpreters
were more energy efficient as forementioned, and the more precise
equivalence detection algorithms were not much slower to run.

\subsection{Team working}

This subsection intentionally left blank.

\end{document}